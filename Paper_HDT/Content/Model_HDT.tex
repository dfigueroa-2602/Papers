% !TEX root = Paper.tex
\subsection{Series Converter}

The series converter dynamics are given by the following state-space equations:

\begin{equation}
    \dot{x}_s =
    \underbrace{
    \begin{bmatrix}
        -\frac{R_{fs}}{L_{fs}} & -\frac{1}{L_{fs}} \\
        \frac{1}{C_{fs}} & 0
    \end{bmatrix}
    }_{\mathbf{A}_s}
    x_s +
    \underbrace{
    \begin{bmatrix}
        \frac{1}{L_{fs}}\\
        0
    \end{bmatrix}
    }_{\mathbf{B}_s}
    u_s +
    \underbrace{
    \begin{bmatrix}
        0 \\
        \frac{1}{C_{fs}}
    \end{bmatrix}
    }_{\mathbf{P}_{ig}}
    i_g^{\alpha\beta} \label{eq:SeriesConverter_Dynamics}
\end{equation}
where $x_s = \begin{bmatrix} i_{fs}^{\alpha\beta} & v_{cs}^{\alpha\beta} \end{bmatrix}^T$ is the state vector, $u_s = v_s^{\alpha\beta}$ is the control input, and $i_g^{\alpha\beta}$ is the grid current. The parameters $R_{fs}$, $L_{fs}$, and $C_{fs}$ are the series converter filter resistance, inductance, and capacitance respectively.

\subsection{Parallel Converter}

In the same way, the parallel converter dynamics are given by the following state-space equations:

\begin{align}
    \begin{aligned}
        \dot{x}_p &=
        \underbrace{
        \begin{bmatrix}
            -\frac{R_{fp}}{L_{fp}} & -\frac{1}{L_{fp}}\\
            \frac{1}{C_{fp}} & 0
        \end{bmatrix}
        }_{\mathbf{A}_p}
        x_p +
        \underbrace{
        \begin{bmatrix}
            \frac{1}{L_{fp}}\\
            0
        \end{bmatrix}
        }_{\mathbf{B}_p}
        u_p\\
        &+
        \underbrace{
        \begin{bmatrix}
            0\\
            -\frac{1}{C_{fp}}
        \end{bmatrix}
        }_{\mathbf{P}_{iY}}
        i_Y^{\alpha\beta}
        +
        \underbrace{
        \begin{bmatrix}
            0\\
            -\frac{1}{C_{fp}}
        \end{bmatrix}
        }_{\mathbf{P}_{iL}}
        i_L^{\alpha\beta} \label{eq:ParallelConverter_Dynamics}
    \end{aligned}
\end{align}
where $x_p = \begin{bmatrix} i_{fp}^{\alpha\beta} & v_{cp}^{\alpha\beta} \end{bmatrix}^T$, $u_p = v_p^{\alpha\beta}$, $i_Y^{\alpha\beta}$ is the transformer secondary side current, and $i_L^{\alpha\beta}$ is the load current. The parameters $R_{fp}$, $L_{fp}$, and $C_{fp}$ are the parallel converter filter resistance, inductance, and capacitance respectively.

\subsection{Distribution Transformer}

The transformer is connected in $\Delta-Y$ configuration, with the series converter (through the coupling transformer) connected to the $\Delta$ side, and the parallel converter connected to the $Y$ side. The $Y$ side has its neutral point grounded. The transformer equations, in a matricial form, are given by:
\begin{align}
    \begin{aligned}
        v_Y^{abc} &= \dfrac{1}{N_{DT}}
        \underbrace{
        \begin{bmatrix}
            1 & -1 & 0 \\
            0 & 1 & -1 \\
            -1 & 0 & 1
        \end{bmatrix}
        }_{\mathbf{K_T}'}
        v_{\Delta}^{abc}\\
        v_Y^{abc} &= \frac{1}{N_{DT}} \mathbf{K_T}'\, v_{\Delta}^{abc} \label{eq:Delta_Y_Transformation}
    \end{aligned}
\end{align}
where $N_{DT}$ is the transformer turns ratio, $v_Y^{abc}$ is the transformer $Y$ side voltage, and $v_{\Delta}^{abc}$ is the transformer $\Delta$ side voltage. Converting \cref{eq:Delta_Y_Transformation} to $\alpha\beta$ coordinates using the Clarke transformation gives:
\begin{equation}
    v_Y^{\alpha\beta} = \dfrac{1}{N_{DT}} \mathbf{K_{T\alpha\beta}}'\, v_{\Delta}^{\alpha\beta}
\end{equation}
where the matrix that models the $\Delta-Y$ transformer is given by:
\begin{equation}
    \mathbf{K_{T\alpha\beta}}' = \begin{bmatrix} \frac{2}{3} & -\frac{\sqrt{3}}{2}\\ \frac{\sqrt{3}}{2} & -\frac{3}{2} \end{bmatrix}
\end{equation}
The dynamics of the transformer are modeled as a series impedance referred to the $Y$ side and its state-space equation is given by:
\begin{equation}
    \dfrac{d\,i_Y^{\alpha\beta}}{d\,t} = -\dfrac{R_Y}{L_Y}i_Y^{\alpha\beta} - \dfrac{1}{L_Y}v_{cp}^{\alpha\beta} + \dfrac{1}{L_Y}N_{DT}\mathbf{K_{T\alpha\beta}}'\, v_{\Delta}^{\alpha\beta} \label{eq:Transformer_Dynamics}
\end{equation}
where $R_Y$ and $L_Y$ are the transformer series resistance and leakage inductance respectively. Since the $\Delta$ side voltage is the sum of the grid voltage and the series converter filter capacitor voltage, it can be expressed as:
\begin{equation}
    v_{\Delta}^{\alpha\beta} = N_{CT}v_{cs}^{\alpha\beta} + v_g^{\alpha\beta}
\end{equation}
where $N_{CT}$ is the coupling transformer turns ratio. Substituting this into \cref{eq:Transformer_Dynamics} gives:
\begin{align}
    \begin{aligned}
        \dfrac{d\,i_Y^{\alpha\beta}}{d\,t} &= -\dfrac{R_Y}{L_Y}i_Y^{\alpha\beta} - \dfrac{1}{L_Y}v_{cp}^{\alpha\beta}\\
        &+ \dfrac{1}{L_Y}N_{DT}\mathbf{K_{T\alpha\beta}}'\left(N_{CT}v_{cs}^{\alpha\beta} + v_g^{\alpha\beta}\right) \label{eq:Transformer_Dynamics_Final}
    \end{aligned}
\end{align}

\subsection{Overall HDT Model}
The overall HDT model is obtained by combining the series converter, parallel converter, and transformer state-space equations given in \eqref{eq:SeriesConverter_Dynamics}, \eqref{eq:ParallelConverter_Dynamics}, and \eqref{eq:Transformer_Dynamics_Final} respectively. The combined state-space equations are given by:

\begin{align}
    \begin{aligned}
        \dfrac{d}{dt}
        \begin{bmatrix}
            x_s\\
            x_p
        \end{bmatrix}
        &=
        \underbrace{
        \begin{bmatrix}
            \mathbf{A}_s & \mathbf{P}_{iY}\mathbf{M}_p \\
            \mathbf{P}_{vc}\mathbf{M}_s & \mathbf{A}_p
        \end{bmatrix}
        }_{\mathbf{A}}
        \underbrace{
        \begin{bmatrix}
            x_s\\
            x_p
        \end{bmatrix}
        }_{x}
        +
        \underbrace{
        \begin{bmatrix}
            \mathbf{B}_s & \mathbf{0} \\
            \mathbf{0} & \mathbf{B}_p
        \end{bmatrix}
        }_{\mathbf{B}}
        \underbrace{
        \begin{bmatrix}
            u_s\\
            u_p 
        \end{bmatrix}
        }_{u}
        \\
        &+
        \underbrace{
        \begin{bmatrix}
            \mathbf{0}\\
            \mathbf{P}_{vg}
        \end{bmatrix}
        }_{\mathbf{P}_{vg}}
        v_g
        +
        \underbrace{
        \begin{bmatrix}
            \mathbf{0}\\
            \mathbf{P}_{iL}
        \end{bmatrix}
        }_{\mathbf{P}_{iL}}
        i_L\\
        \dfrac{d\, x(t)}{dt} &= \mathbf{A}x(t) + \mathbf{B}u(t) + \mathbf{P}_{vg}v_g(t) + \mathbf{P}_{iL}i_L(t)\\
        y(t) &= \mathbf{C}\,x(t)
    \end{aligned}
\end{align}
where the matrices $\mathbf{M}_p = \begin{bmatrix}\mathbf{0} & \mathbf{I} & \mathbf{0}\end{bmatrix}$ and $\mathbf{M}_s = \begin{bmatrix}\mathbf{0} & \mathbf{I}\end{bmatrix}$ are used to select the appropriate states from the parallel and series converter state vectors respectively.

The HDT system is discretized using a zero-order hold with a sampling time of $T_s = 5\,\mu s$. The discrete-time state-space model is given by:
\begin{align}
    \begin{aligned}
        x[k+1] &= \mathbf{A}_d x[k] + \mathbf{B}_d u[k] + \mathbf{P}_{vg,d} v_g[k] + \mathbf{P}_{iL,d} i_L[k]\\
        y[k] &= \mathbf{C} x[k]
    \end{aligned}
\end{align}
where $\mathbf{A}_d = e^{\mathbf{A}T_s}$, $\mathbf{B}_d = \int_0^{T_s} e^{\mathbf{A}\tau} d\tau \mathbf{B}$, $\mathbf{P}_{vg,d} = \int_0^{T_s} e^{\mathbf{A}\tau} d\tau \mathbf{P}_{vg}$, $\mathbf{P}_{iL,d} = \int_0^{T_s} e^{\mathbf{A}\tau} d\tau \mathbf{P}_{iL}$, and $\mathbf{C} = \mathbb{I}$.

In most of the applications, there is a delay of one sampling period between the calculation of the control input and its application to the system. To account for this delay, the discrete-time state-space model is augmented with a new state representing the previous control input:
\begin{equation}
    m_k = u_{k - 1}
\end{equation}
This can be expressed in state-space form as:
\begin{align}
    \begin{aligned}
        \begin{bmatrix}
            x_{k + 1}\\
            m_{k + 1}
        \end{bmatrix}
        &=
        \underbrace{
        \begin{bmatrix}
            \mathbf{A}_d & \mathbf{B}_d \\
            \mathbf{0} & \mathbf{0}
        \end{bmatrix}
        }_{\mathbf{A}_{d,\text{delay}}}
        \begin{bmatrix}
            x_k\\
            m_k
        \end{bmatrix}
        +
        \underbrace{
        \begin{bmatrix}
            \mathbf{0}\\
            \mathbf{I}
        \end{bmatrix}
        }_{\mathbf{B}_{d,\text{delay}}}
        u_k
    \end{aligned}
\end{align}