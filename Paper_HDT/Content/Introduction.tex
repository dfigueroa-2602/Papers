% !TEX root = Paper.tex
\IEEEPARstart{T}{he} increasing penetration of renewable energy sources in the electrical grid has led to a significant rise in the use of power electronic converters. These converters are essential for integrating RES into the grid, as they facilitate the conversion of DC power generated by sources like solar panels and wind turbines into AC power compatible with the grid~\cite{naderiDynamicModelingStability2023}. However, the widespread use of power electronic converters has also introduced several challenges related to power quality, such as the injection of harmonics and non-linear loads, which can lead to voltage distortions and other issues in the electrical grid~\cite{najafzadehRecentContributionsFuture2021,sepasiPowerQualityMicrogrids2023}.

There are many solutions that have been proposed to address the power quality issues in the grid, such as static compensators (STATCOMs)~\cite{engelbrechtSTATCOMTechnologyEvolution2023}, dynamic voltage restorers (DVRs)~\cite{kandilControlOperationDynamic2020}, active power filters (APFs)~\cite{mishraPSOGWOOptimizedFractional2020}, unified power quality conditioners (UPQCs)~\cite{fujitaUnifiedPowerQuality1998} and the solid-state transformers (SSTs)~\cite{huberApplicabilitySolidStateTransformers2019}. These devices have the ability to mitigate most of the power quality issues mentioned above, while also providing galvanic isolation and voltage transformation. However, the high cost and complexity of SSTs have limited their widespread adoption in the distribution grid, and they also do not provide the same short-circuit current capability as traditional distribution transformers (DTs)~\cite{carrenoConfigurationsPowerTopologies2021}.

For this reason, the hybrid distribution transformer (HDT) emerges as a promising solution to address the disadvantages of SSTs while still providing advanced power-quality functionalities.
The HDT is a power electronic transformer that combines the functions of a traditional distribution transformer with those of power electronic converters~\cite{matelskiBadaniaEksperymentalneTransformatora2023,liuSimplifiedDynamicModels2025}. Many HDT configurations have been proposed in the literature, and consequently, classifications have been proposed~\cite{carrenoConfigurationsPowerTopologies2021}. One classification is based on the energy source of the converter, i.e., whether the energy is obtained from a capacitor/battery, the primary or secondary side of the DT, or an auxiliary winding. Another classification is based on how the converters inject energy into the system, i.e., whether they are connected in series or in parallel with the DT. 

Several control strategies have been proposed for the HDT in the literature, including finite control set model predictive control (FCS-MPC)~\cite{costaFourlegMatrixConverter2022}, decoupled control strategies such as the compound controller~\cite{liuCompoundControlSystem2020}, quasi-proportional controllers~\cite{liuQuasiProportionalResonantControlHybrid2022}, and the separated multi-resonant state-feedback controller~\cite{carrenoStateFeedbackControlHybrid2024}. While these approaches have demonstrated good performance, they often rely on heuristic parameter tuning or treat the resonant compensation separately from the state-feedback framework, which complicates the design and hinders systematic optimization. This limitation naturally motivates the use of formal optimization techniques for systematic controller tuning.

In this context, many meta-heuristic algorithms have been proposed in the literature, such as genetic algorithms (GA), differential evolution (DE), artificial bee colony (ABC), grey wolf optimization (GWO), and whale optimization (WOA)~\cite{kavehApplicationMetaHeuristicAlgorithms2023}. Among these, particle swarm optimization (PSO) was selected in this work because it has consistently demonstrated superior convergence speed and computational efficiency for tuning LQR-based state-feedback controllers in power electronic systems~\cite{ufnalskiParticleSwarmOptimization2015}. Prior studies show that PSO achieves fast convergence, requires minimal hyperparameter adjustment, and performs reliably in continuous search spaces representative of LQR weight selection~\cite{chenFeedbackLinearizedOptimal2021}, making it particularly suitable for the optimization problem addressed in this paper.

Building on these observations, this paper proposes a unified control strategy based on state-feedback control for the HDT, where resonant states are incorporated only at a single natural frequency. The proposed strategy achieves zero steady-state error for sinusoidal references and disturbances, accounts for the inherent delay of real digital controllers, and uses PSO to automatically tune the state-feedback gains according to a cost function that balances transient and steady-state performance. The effectiveness of the proposed method is demonstrated through simulation results under various operating conditions.