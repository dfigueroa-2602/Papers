% !TEX root = Paper.tex

In this section, the simulation results of the proposed control strategy are presented. The simulations are performed using MATLAB/Simulink, and the system parameters are listed in Table \ref{tab:params}. The proposed control strategy is tested under grid unbalanced swell, load impact, load unbalance, and non-linear load conditions.

\begin{table}[h!]
    \centering
    \caption{System Parameters}
    \begin{tabular}{|c|c|c|}
        \hline
        Parameter & Variable & Value \\
        \hline
        \hline
        Grid Voltage & $V_{g}$ & 10 kV \\
        Nominal Converter Voltage & $V_s$ & 400 V \\
        DC Link Voltage & $V_{DC}$ & 700 V \\
        Grid Frequency & $f_e$ & 50 Hz \\
        Series Converter Filter Inductance & $L_{fs}$ & 200 mH \\
        Series Converter Filter Resistance & $R_{fs}$ & $100 m\Omega$ \\
        Series Converter Filter Capacitance & $C_{fs}$ & 12 µF \\
        Parallel Converter Filter Inductance & $L_{fp}$ & 200 mH \\
        Parallel Converter Filter Resistance & $R_{fp}$ & $100 m\Omega$ \\
        Parallel Converter Filter Capacitance & $C_{fp}$ & 12 µF \\
        Transformer Dispersion Inductance & $L_Y$ & 100 µH \\
        Transformer Series Resistance & $R_Y$ & $5\ m\Omega$\\
        Coupling Transformer Turns Ratio & $N_{ct}$ & $2.5$ \\
        Distribution Transformer Turns Ratio & $N_{DT}$ & $V_s/(V_g\sqrt{3})$ \\
        Converters Switching Frequency & $f_{\text{sw}}$ & 20 kHz \\
        Control Sampling Time & $T_s$ & 5 µs \\
        \hline
    \end{tabular}
    \label{tab:params}
\end{table}

Each of the load impacts are comprised by $47\ \Omega$ resistive load per phase, $2\ k\Omega$ resistive load in phase $a$ for the unbalanced load, and a three-phase diode bridge rectifier with a $47\ \Omega$ resistive load at its output for the non-linear load. The simulation results are shown in \Cref{fig:Sim_results}, where (a) shows the grid voltage $v_g^{abc}$, (b) shows the load current $i_L^{abc}$, (c) shows the parallel converter capacitor voltage $v_{cp}^{abc}$, and (d) shows the transformer secondary side current $i_Y^{abc}$.

\subsection{Grid Voltage Unbalanced Swell Compensation}

In the instant $t=10\ ms$ until the instant $t = 260\ ms$ as it can be seen in the \Cref{fig:Sim_results}.(a). The proposed control strategy effectively compensates for the unbalanced swell, meaning that the parallel inverter injects the necessary voltage to maintain a balanced transformer secondary side current, as shown in \Cref{fig:Sim_results}.(d) for the transformer secondary side current and in \Cref{fig:Sim_results}.(c) for the parallel inverter voltage. 

\subsection{Load Impact and Unbalanced Load Compensation}

At $t=60\ ms$, a three-phase balanced load impact is applied, and at $t=100\ ms$, a three-phase unbalanced load is applied, where the resistor of the phase $a$ is $2\ k\Omega$, as shown in \Cref{fig:Sim_results}.(b). To keep a balanced transformer secondary side current, the parallel converter injects the needed current, giving the desired performance, as shown in \Cref{fig:Sim_results}.(d)

\subsection{Non-linear Load Compensation}

Following the load impacts, at $t=140\ ms$, a non-linear load is applied, as it shown in the \Cref{fig:Sim_results}.(b), which consists of a three-phase diode bridge rectifier with a resistive load of $47\ \Omega$ at its output. The parallel converter compensates for the harmonics injected by the non-linear load, giving the secondary side currents that are shown in the \Cref{fig:Sim_results}.(d).

\subsection{Grid Harmonics Compensation}

Finally, at $t=200\ ms$, a third-order harmonic is added to the grid voltage, as shown in \Cref{fig:Sim_results}.(a). The parallel converter compensates for the harmonic distortion, giving the series converter capacitor voltage shown in \Cref{fig:Sim_results}.(c).